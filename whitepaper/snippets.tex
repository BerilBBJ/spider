
\CONTENT{Describe the scope of the project, refer to previous work (e.g., contrast with Nicolas Christin’s work on Silk Road as well as the work where they ran their own relays to collect content). Clearly state our contributions. Summarize the methodology and results.}
\NOTE{Introduce visible vs invisible part of the darknet}
Visible part: the one that can be accessed using the Tor network without any further information needed. Invisible part: the ones that require authentication or any other knowledge to be found. We classify what a normal Tor user can find on the network. Of course, this does not cover all possible usages of the Tor network, but as more users access the tor network, we can be sure that most of them will stay in the visible part (therefor this is relevant)

Ok, here a short summary and why we are making something different than the linked articles/blogs: \\
*https://www.hyperiongray.com/dark-web-map/* : Displays 6.6k websites in a "map". However, the map is sorted by structural similarity and many of the displayed pages are either offline (E.g. taken offline by law enforcement, by the hoster itself (only displays a notice) or are not yet setup ("I congratulate you! You have made a site on the dark side (onion)"). I think it's clear that this is not what we are doing\\
*TERBIUM LABS* Work (I assume 2016 from the files name, but is not stated clearly): Reviewed a sample of 400 URLs, collected by their crawler, classified into one of the following categories\\
•Legal\\
•Explicit\\
•Drugs\\
•Pharmaceuticals\\
•Fraud\\
•Multiple Categories (Illicit)\\
•Falsified Documents & Counterfeits\\
•Exploitation\\
•Hacking & Exploits\\
•Weapons\\
•Extremism\\
•Weapons of Mass Destruction\\
•Other Illicit Activity \\
•Unknown/Site Down\\
•Downloadable File\\
They come to the conclusion that the darknet is mainly legal, however their work is based on a very small sample size\\
Interesting sections for us: "Handling Illegal Content and Exploitation", "Why we used a Random Sample"\\
*Empirical analysis of Tor Hidden Services* (ET Information Security, 2015): They collected their data by running a large number of Tor nodes for 6 months and collected 80'000 hidden services that way. They did a size estimation and a content categorization, results seem to approximately match up with \textit{Content and popularity analysis ofTor hidden services}. Since they operated Tor nodes, they also collected which category produces what amount of traffic. They have a small section on Darknet connectivity and found that 59\% of the pages do only link to itself.
Topic pretty similar to ours, but the approach is different \\   
*Content and popularity analysis of Tor hidden services* (2014): They used a flaw in the Tor Protocol to gather their data. That way they collected 39800 pages. They propose how one can deanonymise clients of HS, they did a content analysis as well and they looked at the open ports of the found hosts. This enabled them to give an overview of which software is used to deliver the HS.
Topic also pretty similar to ours, but again, our approach is differen.

Conclusion: \textit{Empirical analysis of Tor Hidden Services} and \textit{Content and popularity analysis of Tor hidden services} do a pretty similar analysis on their data, but\\
a) The papers are at least 3 years old \\
b) They either cracked the Tor protocol or used Tor nodes to discover their data. We do our work on the "visible" part of the darknet and therefor do not acces any data the is not meant to be public (or only by human error - however, we are not the ones to blame for)\\
c) As far as I know there does not exist any work that has tried to apply a scraper to the darknet for classification (except terbium labs, but this is commercial and there paper only looked at 400 random samples by manual inspection. Our proposed technique allows for continuously scraping and classifying the content, which also allows to catch temporal changes in the topology of the network itself)\\
In short: our contribution: Tested the effectiveness of scraping, a non intrusive, non-deanonymizing technique to gather data from the Tor network. Further it would allow to capture temporal changes in the network. Since we can only look at publicly available pages, we do not compromise any data security or similar.



\section{Editorial Policy}

\subsection{Choice of reviewers}

The Editor responsible for a given area of physics, turns to experts of the subject for opinion. Research articles and communications are reviewed by minimum two reviewers, review papers by at least three.

\subsection{Suggestions from authors}

Authors are requested to suggest persons competent to review their manuscript. However, please note that this will be treated only as a suggestion, the final selection of reviewers is exclusively the Editor's decision. The reviewers remain anonymous in any case.

The Editor is fully responsible for decision about the manuscript. The final decision, whether to accept or reject a paper, rests with him/her. The Managing Editor only communicates the final decision and informs the author about further processing.

\subsection{Revised manuscript submission}

When revision of a manuscript is requested, authors should return the revised version of their manuscript as soon as possible. Prompt action may ensure fast publication, if the paper is finally accepted for publication in .... If it is the first revision of an article authors need to return their revised manuscript within 60 days. If it is the second revision authors need to return their revised manuscript within 14 days. If these deadlines are no met, and no specific arrangements for completion have been made with the Editor, the manuscript will be treated as a new one and will receive a new identification code along with a new registration date.

\subsection{Final proofreading}

Authors will receive a PDF file with the edited version of their manuscript for final proofreading. This is the last opportunity to view an article before its publication on the journal's web site. No changes or modifications can be introduced once it is published. Thus authors are requested to check their proof pages carefully against manuscript within 3 working days and prepare a separate document containing list of all the changes that should be introduced. Authors are sometimes asked to provide additional comments and explanations in response to remarks and queries from the language and technical editors. In case the authors do not deliver the list of corrections to proofs in the requested time the manuscript will be published as is.

 

\subsection{Reprints}

Because the journal is published in an Open Access model, and has no printed version, the authors receive no reprints.

\subsection{Erratum}

If any errors are detected in the published material they should be reported to the Managing Editor. The corresponding authors should send appropriate corrected material to the Managing Editor via email. This material will be considered for publication in form of erratum in the earliest available issue of ....

\subsection{Copyright  }

All authors retain copyright, unless -- due to their local circumstances -- their work is not copyrighted. The non-commercial use of each article will be governed by the Creative Commons Attribution-NonCommercial-NoDerivs license. The corresponding author grants De Gruyter Open the exclusive license to commercial use of the article, by signing the License to Publish. Scanned copy of license should be sent by e-mail to the Managing Editor of the journal, as soon as possible.

%% ###################################################################

\section{Paper writing guide}

\subsection{Paper elements}

\begin{enumerate}
\item title page with:
    \begin{enumerate}
    \item title (short title),
    \item full name(s) of author(s),
    \item name and address of workplace(s),
    \item personal e-mail address(es),
    \end{enumerate}
\item abstract,
\item up-to five keywords,
\item text,
\item reference lists.
\end{enumerate}


\subsubsection{Abstract}

An abstract must accompany every article. It should be a brief summary of the significant items of the main paper. An abstract should give concise information about the content of the core idea of your paper. It should be informative and not only present the general scope of the paper but also indicate the main results and conclusions. An abstract should not normally exceed 200 words. It should not contain literature citations or allusions to the tables or illustrations. All non-standard symbols and abbreviations should be defined.

In combination with the title and key-words, the abstract is an indicator of the content of the paper. Authors should remember that on-line systems rely heavily on the content of titles and abstracts to identify articles in electronic bibliographic databases and search engines. They are therefore requested to take great care in preparing these elements.


\subsubsection{Text}

\paragraph{General rules for writing}
\begin{itemize}
\item use simple and declarative sentences, avoid long sentences, in which the meaning may be lost by complicated construction;
\item be concise, avoid idle words;
\item make your argumentation complete; use commonly understood terms; define all non-standard symbols and abbreviations when you introduce them;
\item explain all acronyms and abbreviations when they first appear in the text;
\item use all units consistently throughout the article;
\item be self-critical as you review your drafts.
\end{itemize}

\paragraph{Structure of a paper}
    Research papers and review articles should follow a strict structure. Generally a standard scientific paper is divided into:
\begin{itemize}
\item introduction: you present the subject of your paper clearly, you indicate the scope of the subject, you present the goals of your paper and finally the organization of your paper;
\item main text: you present all important elements of your scientific message;
\item conclusion: you summarize your paper.
\end{itemize}


    Experimental part and/or calculations should be presented in sufficient details to enable reader to repeat the original work.



\paragraph{Footnotes/End-notes/Acknowledgments}
We encourage authors to restrict the use of footnotes. If necessary, please make end-notes rather than footnotes. Allowable footnotes/end-notes may include:

\begin{itemize}
\item the designation of the corresponding author of the paper;
\item the current address of an author (if different from that shown in the affiliation);
\item traditional footnote content.
\end{itemize}

\paragraph{Tables}
    Authors should use tables only to achieve concise presentation, or where the information cannot be given satisfactorily in other ways. Tables should be numbered consecutively using Arabic numerals and referred to in the text by number. Each table should have an explanatory caption which should be as concise as possible.



\paragraph{Figures}
    Authors may use line diagrams and photographs to illustrate theses from their text. The figures should be clear, easy to read and of good quality. Styles and fonts should match those in the main body of the article. All figures must be mentioned in the text in consecutive order and be numbered with Arabic numerals. 

\begin{figure}
\framebox[4cm]{\Huge Figure 1}
\caption{A figure caption should be placed {\bf below} the figure.\label{fig1}}
\end{figure}

\begin{figure}
\framebox[4cm]{\Huge Figure 2}
\caption{A figure caption for Fig. \ref{fig2}.\label{fig2}}
\end{figure}

\paragraph{Typesetting}
    Type main text in roman (upright) font. The chemical symbols and compounds, units of measure, most multi-letter operators and functions should are written in roman upright as well. The variables, constants, symbols for particles, most single-letter operators, axes and planes, channels, types (e.g., n, p), bands, geometric points, angles, lines, chemical prefixes, symmetry designations, transitions, critical points, color centers, quantum-state symbols in spectroscopy, and most single-letter abbreviations should be written in roman italic. Boldface roman type is reserved for indicating vectors and in some special cases matrices. 


\paragraph{Mathematical symbols}
    The multiplication signs are reserved for a vector product ($\mathbf{A}\times\mathbf{B}$) and simple dot product ($\mathbf{A}\cdot\mathbf{B}$). The only exception are numbers expressed in scientific notation ($9.7\times 10^3$ MeV).


\paragraph{Units}
    Units and dimensions should be expressed according to the metric system and SI units. This system is based on: meter (m), second (s), kilogram (kg), ampere (A), kelvin (K), mole (mol), and candela (cd). Most units are spaced off from the number, e.g. 12 mV. The only exceptions are:
\begin{center}
    1\%, 1\textperthousand, 1\textdegree C, 1\textdegree, 1', 1".
\end{center}

    Decimal multiples or sub-multiples of units are indicated by the use of prefixes

\begin{center}
    \textmu=$10^{-6}$, m$=10^{-3}$, c$=10^{-2}$, d$=10^{-1}$,
    da$=10^1$, h$=10^2$, k$=10^3$, M$=10^6$, G$=10^9$, {\em etc}.
\end{center}

    Compound units are written as
\begin{center}
    4221.9 J kg$^{-1}$ K$^{-1}$ or 4221.9 J/(kg K),
\end{center}
    with a thin space between unit parts.


    Authors should indicate precisely in the main text {\bf where tables and figures should be inserted}, if these elements are given at the end in the original version of the manuscript (or supplied in separate files).
    If this information is not provided along with the manuscript, we will assume that the figures and/or tables should be insert at the closest position to first reference to them in the published paper.

\paragraph{Multimedia and images}
    Authors can attach files in most popular formats, including (for example):
\begin{itemize}
\item images in BMP, GIF, JPEG formats,
\item multimedia files in MPEG or AVI formats.
\end{itemize}

However please keep to file types that are read by standard media players (e.g. RealPlayer, Quicktime, Windows Media Player) and/or standard office applications (Adobe Acrobat Reader, Microsoft Office etc.).

    Your attachments may be accessible through links to external locations or to our internal locations (if you choose the second option, please remember to send us your attachments).

    Please remember that your images, video and animation clips are intended for Internet use and we need to consider the needs of users with slow Internet connections. Please try to minimize file sizes by using a lower resolution or number of colors for images and animations (as long as the material is still clear). To help you in formatting your images (including tables and figures) or multimedia files, please submit your paper with separate attachments, which are used in your paper.

\paragraph{English language}
 Journal     is published only in English. Make sure that your manuscripts are clearly and grammatically written. Please note that authors who are not native-speakers of English can be provided with help in rewriting their contribution in correct English. Try to prepare your manuscript in an easily readable style; this will help avoid severe misunderstandings which might lead to rejection of the paper.

\subsubsection{Reference list}

A complete reference should give the reader enough information to find the relevant article. All authors (unless there are six or more) should be named in the citation. If there are six or more, list the name of the first one followed by ``et al''. Please pay particular attention to spelling, capitalization and punctuation here. Completeness of references is the responsibility of the authors. A complete reference should comprise the following:

\paragraph{Reference to an article in a journal}
Elements to cite:
Author's Initials. Surname, -- if more authors, see examples below,
Title of journal -- abbreviated according to the ISI standards\footnote{ http://images.isiknowledge.com/WOK46/help/WOS/0-9\_abrvjt.html},
volume number, page or article number (year of publication).
Please supply DOI or URL for e-version of the papers.
See Refs. \cite{journal-1, journal-2, journal-3, journal-4, journal-5, journal-6, journal-7, journal-8} for example.

\paragraph{Reference to a book}
Elements to cite:
Author's Initials. Surname,
Title,
Edition -- if not the first
(Publisher, Place of publication, Year of publication)
\cite{book}.


\paragraph{Reference to a part/chapter book}
Elements to cite:
Author's Initials. Surname,
In: Editor's Initials. Editor's Surname (Ed.),
Book Title,
Edition -- if not the first,
(Publisher, Place of publication, Year of publication)
page number \cite{chapter}.


\paragraph{Reference to a preprint}
Elements to cite:
Author's Initials. Surname,
arXiv:preprint-number and version \cite{arxiv-1,arxiv-2}.

\paragraph{Reference to a conference proceedings}
Elements to cite:
Author's Initials. Surname,
In: Editor's Initials. Editor's Surname (Ed.),
Conference,
date, place (town and country) of conference
(Publisher, place of publication, year of publication)
page number \cite{proceedings}.


\paragraph{Reference to a thesis}
Elements to cite:
Author's Initials. Surname,
D.Sc./Ph.D./M.Sc./B.Sc. thesis,
University,
(town, country, year of publication) \cite{thesis}.


\paragraph{Reference to an article in a newspaper}
Elements to cite:
Author's Initials. Surname,
Newspaper Title,
date of publication,
page number \cite{newspaper-1,newspaper-2}.


\paragraph{Reference to a patent}
Elements to cite:
Originator,
Series designation which may include full date \cite{patent}.


\paragraph{Reference to a standard}
Elements to cite:
Standard symbol and number,
Title \cite{standard-1,standard-2}.

Please add language of publication for materials which are not written in English. Indicate materials accepting for publications by adding ``(in press)''. Please avoid references to unpublished materials, private communication and web pages.

You should make sure the information is correct so that the linking reference service may link abstracts electronically. For the same reason please separate each reference from the others.

Before submitting your article, please ensure you have checked your paper for any relevant references you may have missed.

\subsubsection{Submission formats}

Manuscripts for ... should be submitted in the \LaTeX ~format with figures in EPS, PDF or PNG format. Authors are strongly encouraged to register their manuscript in arXiv preprint server and submit it to our Editorial Manager using arXiv's paper ID.

\subsubsection{Supplementary data}

You can also submit any supplementary data files as well. These may include long tables (in HTML or plain TXT format) or movies (preferably in AVI format).

\begin{thebibliography}{99}
\bibitem{}
\bibitem{journal-1} A.~P.~Raposo, H.~J.~Weber, D.~E.~Alvarez--Castillo, M.~Kirchbach, Cent. Eur. J. Phys. 5, 253 (2007)
\bibitem{journal-2} J.~Barth et al. (SAPHIR Collaboration), Phys. Lett. B 572, 127 (2003)
\bibitem{journal-3} S.~Chekanov et al., Eur. Phys. J. C 51, 289 (2007)
\bibitem{journal-4} K.~Malarz, Postepy Fizyki 57, 235 (2006) (in Polish)
\bibitem{journal-5} G.~Meng, Cent. Eur. J. Phys., DOI:10.2478/s11534-007-0038-1
\bibitem{journal-6} R.~Hegselmann, U.~Krause, Journal of Artificial Societies and Social Simulation (2006), http://jasss.soc.surrey.ac.uk/9/3/10.html
\bibitem{journal-7} A.~Dybala, Cent. Eur. J. Chem. (in press)
\bibitem{journal-8} A.~Dybala, Przeglad chemiczny (in Polish, in press)
\bibitem{book} M.~Lister, Fundamentals of Operating Systems, 3rd edition (Springer-Verlag, New York, 1984)
\bibitem{chapter} C.~K.~Clenshaw, K.~Lord, In: B.~K.~P.~Scaife (Ed.), Studies in Numerical Analysis (Academic Press, London and New York, 1974) 95
\bibitem{arxiv-1} M.~Majewski, K.~Malarz, arXiv:cond-mat/0609635v2 [cond-mat.stat-mech]
\bibitem{arxiv-2} J.~A.~C.~E.~Solano, arXiv:0707.1343v1 [astro-ph]
\bibitem{proceedings} A.~Kaczanowski, K.~Malarz, K.~Kulakowski, In: T.~E.~Simos (Ed.), International Conference of Computational Methods in Science and Engineering, Sep. 12-16, 2003, Kastoria, Greece (World Scientific, Singapore 2003) 258
\bibitem{thesis} A.~J.~Agutter, Ph.D. thesis, Edinburgh University (Edinburgh, UK, 1995)
\bibitem{newspaper-1} A.~Sherwin, The Times, Jul. 13, 2007, 1
\bibitem{newspaper-2} M.~Dzierzanowski, Wprost, Jul. 8, 2007, 18 (in Polish)
\bibitem{patent} Philip Morris Inc., European patent application 0021165 A1, Jan. 7, 1981
\bibitem{standard-1} ISO 2108:1992, Information and documentation --- International standard book numbering (ISBN)
\bibitem{standard-2} ISO/TR 9544:1988, Information processing --- Computer-assisted publishing --- Vocabulary
\end{thebibliography}
